\section{Diskussion}

Die berechnete Aktivierungsenergie mit den unterschiedlichen Berechnungsmethodenund für die zwei Heizraten unterliegt großen Schwankungen. Für die Heizrate von \SI{2.0}{\kelvin\per\second} sind die Werte aus den unterschiedlichen Methoden miteinander verträglich, für \SI{1.5}{\kelvin\per\second} sind die Werte für die Aktivierungsenergie nicht mit den Werten der anderen Heizrate vergleichbar. Alle Ergebnisse liegen weit über der theoretischen Aktivierungsenergie von KBr(Sr), welche \SI{0.66}{\electronvolt} beträgt \cite{KBr-Sr}.\par
Durch den Fehler in der Aktivierungsenergie, wird die Relaxationszeit extrem unterschätzt und liegt in allen Fällen mehrere Größenordnungen unterhalb des Theoriewertes von \SI{4e-14}{\second}.\par
Eine mögliche Ursache für die Abweichungen ist die Annahme einer konstanten Heizrate, die manuell über einen Heizstrom geregelt wurde und so nicht erfüllt war. Auch das Ablesen der Messwerte für den Strom auf einer analogen Skala ist stark fehlerbehaftet. Bewegungen in der Nähe des Messgerätes und das Umschalten der Messskalaführen zu Abweichungen.\par
Weiterhin ist die Vorbereitung der Messung eine Fehlerquelle, wenn zum Beispiel die Probe nicht hinreichend lange oder zu hinreichend hoher Temperatur erhitzt wird, sodass es zu Schwankungen in der Zahl der ausgerichteten Dipole im Feld des Plattenkondensators kommt. Weiterhin muss der Kondensator für die Messung vollständig entladen sein. Ist dies nicht gegeben, kommt es vor allem am Anfang der Messreihe zu Unsicherheiten in der Strommessung.\par
Für eine genauere Bestimmung der Relaxationszeit muss die Bestimmung der Aktivierungsenergie um ein Vielfaches genauer sein. Dazu kann zum Beispiel eine elektronische Regelung der Heizrate verwendet und der Heiz- und Abkühlvorgang der Probe automatisiert werden.

\printbibliography

\section{Anhang}

\begin{table}
  \centering
  \caption{Messwerte für die Heizrate \SI{1.5}{\kelvin\per\minute}}
\begin{tabular}{cc|cc|cc}
        T/C & I/A &T/C  & I/A &T/C & I/A \\
\midrule
 -50.0 &  1.2e-11 & -19.0 & 1.4e-10 & 16.0 &  7.7e-11 \\
 -49.0 &  1.3e-11 & -18.0 & 1.5e-10 & 17.0 &  7.9e-11 \\
 -48.0 & 1.3e-11 &  -17.0 & 1.5e-10 & 18.0 &  8.3e-11 \\
 -47.0 & 1.3e-11 &  -16.0 & 1.5e-10 & 19.0 &  8.6e-11 \\
 -46.0 & 1.3e-11 &  -15.0 & 1.5e-10 & 20.0 &  9.0e-11 \\
 -45.0 & 1.3e-11 &  -14.0 & 1.5e-10 & 21.0 &  9.5e-11 \\
 -44.0 & 1.4e-11 &  -13.0 & 1.4e-10 & 22.0 &  1.1e-10 \\
 -43.0 & 1.4e-11 &  -12.0 & 1.4e-10 & 23.0 &  1.2e-10 \\
 -42.0 & 1.4e-11 &  -11.0 & 1.3e-10 & 24.0 &  1.2e-10 \\
 -41.0 & 1.5e-11 &  -10.0 & 1.2e-10 & 25.0 &  1.3e-10 \\
 -40.0 & 1.5e-11 &   -9.0 & 1.1e-10 & 26.0 &  1.4e-10 \\
 -39.0 & 1.6e-11 &   -8.0 & 1.0e-10 & 27.0 &  1.5e-10 \\
 -38.0 & 1.7e-11 &   -7.0 & 7.3e-11 & 28.0 &  1.5e-10 \\
 -37.0 & 1.8e-11 &   -6.0 & 6.5e-11 & 29.0 &  1.6e-10 \\
 -36.0 & 1.9e-11 &   -5.5 & 6.1e-11 & 30.0 &  1.7e-10 \\
 -35.0 & 2.1e-11 &   -5.0 & 5.7e-11 & 31.0 &  1.8e-10 \\
 -34.0 & 2.3e-11 &   -4.5 & 5.5e-11 & 32.0 &   1.9e-10 \\
 -33.0 & 2.5e-11 &   -4.0 & 5.2e-11 & 33.0 &   1.9e-10\\
 -32.0 & 2.9e-11 &   -3.0 & 4.9e-11 & 34.0 &  2.0e-10 \\
 -31.0 & 3.2e-11 &   -2.0 & 4.7e-11 & 35.0 &  2.0e-10 \\
 -30.0 & 3.7e-11 &   -1.0 & 4.6e-11 & 36.0 &  2.0e-10 \\
 -29.0 & 4.2e-11 &    0.0 & 4.6e-11 & 37.0 &  2.0e-10 \\
 -28.0 & 4.6e-11 &    1.0 & 4.7e-11 & 38.0 &  2.0e-10 \\
 -27.0 & 5.3e-11 &    2.0 & 4.9e-11 & 39.0 &  2.0e-10 \\
 -26.0 & 6.1e-11 &   3.0 &  5.0e-11 & 40.0 &  2.0e-10 \\
 -25.0 & 6.8e-11 &   4.0 &  5.2e-11 & 41.0 &  1.9e-10 \\
 -24.5 & 7.2e-11 &   5.0 &  5.4e-11 & 42.0 &  1.9e-10 \\
 -24.0 & 7.6e-11 &   6.0 &  5.6e-11 & 43.0 &  1.8e-10 \\
 -23.5 & 8.0e-11 &   7.0 &  5.8e-11 & 44.0 &  1.8e-10 \\
 -23.0 & 8.5e-11 &   8.0 &  6.0e-11 & 45.0 &  1.7e-10 \\
 -22.5 & 8.9e-11 &   9.0 &  6.2e-11 & 46.0 &  1.6e-10 \\
 -22.0 & 9.2e-11 &  10.0 &  6.4e-11 & 47.0 &  1.5e-10 \\
 -21.5 & 9.7e-11 &  11.0 &  6.6e-11 & 48.0 &   1.5e-10 \\
 -21.0 & 1.1e-10 &  12.0 &  6.8e-11 & 49.0 &  1.4e-10 \\
 -20.5 & 1.2e-10 &  13.0 &  7.0e-11&  50.0 &  1.4e-10 \\
 -20.0 & 1.3e-10 &  14.0 &  7.2e-11 & & \\
 -19.5 & 1.3e-10 &  15.0 &  7.4e-11 & &
\end{tabular}
\end{table}

\begin{table}
  \centering
  \caption{Messwerte für die Heizrate \SI{2.0}{\kelvin\per\minute}}
\begin{tabular}{cc|cc|cc}
        T/C & I/A &T/C  & I/A &T/C & I/A \\
\midrule
       -62.0 &  6.7e-12 &       -13.0 &  1.9e-10 &        27.0 &  1.8e-10 \\
       -60.0 &  6.8e-12 &       -11.0 &  1.8e-10 &        28.0 &  1.9e-10 \\
       -59.0 &  7.0e-12 &       -10.0 &  1.7e-10 &        29.0 &  2.0e-10 \\
       -57.0 &  7.5e-12 &        -9.0 &  1.5e-10 &        30.0 &  2.2e-10 \\
       -56.0 &  7.7e-12 &        -8.0 &  1.4e-10 &        31.5 &  2.4e-10 \\
       -55.0 &  8.0e-12 &        -7.0 &  1.3e-10 &        32.0 &  2.5e-10 \\
       -54.0 &  8.1e-12 &        -6.0 &  1.1e-10 &        34.0 &  2.8e-10 \\
       -53.0 &  8.3e-12 &        -5.0 &  1.0e-10 &        35.0 &  3.0e-10 \\
       -52.0 &  8.5e-12 &        -4.5 &  8.5e-11 &        36.0 &  3.1e-10 \\
       -50.0 &  8.7e-12 &        -4.0 &  7.8e-11 &        37.0 &  3.3e-10 \\
       -49.0 &  8.9e-12 &        -3.0 &  7.0e-11 &        38.0 &  3.5e-10 \\
       -48.0 &  9.1e-12 &        -2.5 &  6.6e-11 &        39.0 &  3.6e-10 \\
       -47.0 &  9.2e-12 &        -2.0 &  6.4e-11 &        40.0 &  3.8e-10 \\
       -46.0 &  9.4e-12 &        -1.0 &  6.1e-11 &        41.0 &  4.0e-10 \\
       -44.0 &  1.1e-11 &         0.0 &  5.9e-11 &        42.0 &  4.2e-10 \\
       -42.0 &  1.1e-11 &         1.0 &  5.8e-11 &        43.0 &  4.3e-10 \\
       -40.0 &  1.3e-11 &         2.0 &  5.9e-11 &        44.0 &  4.4e-10 \\
       -38.0 &  1.5e-11 &         3.0 &  6.0e-11 &        45.0 &  4.5e-10 \\
       -36.0 &  1.8e-11 &         4.0 &  6.2e-11 &        46.0 &  4.5e-10 \\
       -35.0 &  2.0e-11 &         5.0 &  6.4e-11 &        47.0 &  4.6e-10 \\
       -34.0 &  2.3e-11 &         6.0 &  6.7e-11 &        48.0 &  4.6e-10 \\
       -33.0 &  2.6e-11 &         7.0 &  6.9e-11 &        49.0 &  4.6e-10 \\
       -32.0 &  2.9e-11 &         8.0 &  7.2e-11 &        50.0 &  4.6e-10 \\
       -31.0 &  3.3e-11 &         9.0 &  7.5e-11 &        51.0 &  4.5e-10 \\
       -30.0 &  3.8e-11 &        10.0 &  7.7e-11 &        52.0 &  4.4e-10 \\
       -29.0 &  4.2e-11 &        11.0 &  8.1e-11 &        53.0 &  4.2e-10 \\
       -28.0 &  4.8e-11 &        12.0 &  8.4e-11 &        54.0 &  4.1e-10 \\
       -27.0 &  5.5e-11 &        13.0 &  8.7e-11 &        55.0 &  3.9e-10 \\
       -26.0 &  6.3e-11 &        14.0 &  9.0e-11 &        56.0 &  3.7e-10 \\
       -25.0 &  7.1e-11 &        15.0 &  9.4e-11 &        57.0 &  3.5e-10 \\
       -24.0 &  8.0e-11 &        16.0 &  9.7e-11 &        58.0 &  3.3e-10 \\
       -23.0 &  9.0e-11 &        17.0 &  1.1e-10 &        59.0 &  3.0e-10 \\
       -22.0 &  1.0e-10 &        18.0 &  1.2e-10 &        60.0 &  2.8e-10 \\
       -21.0 &  1.2e-10 &        19.0 &  1.2e-10 &        61.0 &  2.6e-10 \\
       -20.0 &  1.4e-10 &        20.0 &  1.3e-10 &       62.0  &  2.5e-10 \\
       -19.0 &  1.5e-10 &        21.0 &  1.3e-10 &        63.0 &  2.3e-10 \\
       -18.0 &  1.6e-10 &        22.0 &  1.4e-10 &        64.0 &  2.2e-11 \\
       -17.0 &  1.7e-10 &        23.0 &  1.5e-10 &  & \\
       -16.0 &  1.8e-10 &        24.0 &  1.5e-10 &  & \\
       -15.0 &  1.8e-10 &        25.0 &  1.6e-10 &  & \\
       -14.0 &  1.9e-10 &        26.0 &  1.7e-10 &  &
\end{tabular}
\end{table}

