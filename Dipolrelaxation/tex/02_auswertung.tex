\section{Auswertung}

Eine genauere Bestimmung der Aktivierugnsenergie $W$ erfolgt anhand der Gleichung
\begin{equation*}
    F(T) = \frac{\int_T^{T^\ast} i(T^\prime)\mathrm{d}T^\prime}{i(T)}\,.
\end{equation*}
Dazu wird $F(T)$ linear gegen $1/T$ aufgetragen:
\begin{equation*}
    F(T) = a \cdot\frac{1}{T} + b
\end{equation*}
Die Aktivierugnsenergie ist dann
\begin{equation*}
    W = a \cdot k_\text{B}\,.
\end{equation*}
Die Relaxationszeit $\tau_0$ wird mit die Lage des maximalen Stroms $T_\text{max}$ bestimmt, hier fehlt noch eine genauere Beschreibung, wie man zu dieser Gleichung kommt:
\begin{equation}
    \label{eqn:tau0}
    \tau_0 = \frac{k_\text{B}T_\text{max}^2}{Wb}
             \exp\!\left[-\frac{W}{k_\text{B}T_\text{max}} \right]
\end{equation}
