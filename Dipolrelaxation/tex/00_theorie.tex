\section{Theorie}

Durch Messung des Relaxationsstroms, der entsteht, wenn zuvor in einem elektrischen Feld ausgerichtete Dipole zurück in einen statistischen Zustand relaxieren, werden die Aktivierungsenergie und die Relaxationszeit von mit Strontium dotierten Kaliumbromid bestimmt.

\subsection{Permanente Dipole im Ionenkristall}

Durch das Dotieren des einwertigen Ionenkristalls Kaliumbromid mit zweiwertigen $\text{Strontium}^{2+}$-Ionen, entstehen Paare von Strontium-Ionen und Leerstellen, da eine lokale, elektrische Ladungsneutralität energetisch bevorzugt ist. So werden im Kristall permanante Dipole erzeugt. Die Richtungs der Dipole ist gleichverteilt. Die Energie, die aufgebracht werden muss, um die Richtung eines Dipols im Gitter zu verändern beträgt $W$; die Anzahl der Dipole, die diese Energie bei einer Temperatur $T$ aufbringen können ist nach Boltzmannstatistik $\exp \left( -W/k_\text{B}T \right)$. Die Relaxationszeit $\tau$ ist eine Größe, die die mittlere Zeit zwischen zwei Umorientierungen eines Dipols beschreibt. Sie ist temperaturabhängig, nähert sich mit steigender Temperatur dem Wert $\tau_0$ an und lässt sich beschreiben mit
\begin{equation}
  \tau(T) = \tau_0 \exp \left( W/k_\text{B}T \right)\,.
  \label{eq:tau}
\end{equation}

\subsection{Grundlagen des Messverfahrens}

Wird ein solcher Kristalls als Dielektrikum zwischen zwei Kondensatorplatten angebracht und ein elektrisches Feld angelegt, richtet sich ein Anteil der Dipole gemäß dem elektrischen Feld aus, sodass die Richtungsverteilung nicht mehr rein statistisch ist. Der Anteil der ausgerichteten Dipole ist dann näherungsweise
\begin{equation}
  y(p) = \frac{pE}{3k_\text{B}T}\,,
\end{equation}
wobei $p$ der Betrag des Dipolmoments und $E$ die elektrische Feldstärke ist.\par
Durch ein Abkühlen mit Hilfe von flüssigem Stickstoff, erhöht sich die Relaxationszeit, die Anzahl der ausgerichteten Dipole wird jedoch kaum verändert. Die Probe wird dann mit einer konstanten Heizrate $b = \text{d}T/\text{d}t$ aufgeheizt, sodass die zuvor ausgerichteten Dipole wieder eine statistische Verteilung annehmen. Dieser Vorgang ist die Ursache für einen Relaxationsstrom. Durch die Abnahme der Relaxationszeit mit steigender Temperatur, steigt der Strom an, durchläuft ein Maximum und nimmt wieder ab, da sich die Zahl der noch nicht relaxierten Dipole verringert.
Da die Zahl der bei der Temperatur $T$ im Zeitintervall $\text{d}t$ relaxierenden Dipole abhängig von der Zahl der noch vorhandenen, orientierten Dipole ist, mit der inversen Relaxationszeit $\tau(T)$ als Proportionalitätskonstante, ist die Zahl der ausgerichteten Dipole zur Zeit $t$ bzw. bei der Temperatur $T$
\begin{equation}
  N = N_\text{p} \exp{\left( - \int_{t_0}^{t} \frac{\text{d}t^\prime}{\tau(T)} \right)} = N_\text{p} \exp{\left( - \frac{1}{b} \int_{T_0}^{T} \frac{\text{d}T^\prime}{\tau(T^\prime)} \right)}\,.
\end{equation}
Die Depolarisationsstromdichte ist
\begin{equation}
  j(T) = \frac{p^2E}{3k_\text{B}T_\text{p}} \frac{N_\text{p}}{\tau_0} \exp{\left[ -\frac{1}{b\tau_0}\int_{T_0}^{T} \exp{\left( -W / k_\text{B}T^\prime \text{d}T^\prime \right)} \right]} \exp{\left( -W / k_\text{B}T \right)}\,.
  \label{eq:stromdichte}
\end{equation}
Da für den Anfang der Kurve
\begin{equation}
  \int_{T_0}^{T} \exp{\left( -W / k_\text{B}T^\prime \text{d}T^\prime \right)} \approx 0
\end{equation}
gilt, kann die Depolarisationsstromdichte dort mit
\begin{equation}
  j(T) = \frac{p^2E}{3k_\text{B}T_\text{p}} \frac{N_\text{p}}{\tau_0} \int_{T_0}^{T} \exp{\left( -W / k_\text{B}T \right)}
  \label{eq:meth1}
\end{equation}
genähert werden; Eine lineare Ausgleichsrechung an $\ln{j(T)}$ gegen $1/T$ liefert dann die Aktivierungsenergie $W$.\par
Um ein genaueres Ergebnis zu erhalten wird der gesamte Kurvernverlauf berücksichtigt. Da die Änderung der Polarisation einen zur Änderungsrate proportionalen Zusammenhang besitzt und die Relaxationsfrequenz ebenfalls proportional zu der Änderung der Polarisation ist, besteht zwischen dem Strom und der Relaxationszeit der Zusammenhang
\begin{equation}
  \tau(T) = \frac{\int_{T}^{\infty} i(T^\prime) \text{d}T^\prime}{b \cdot i(T)}\,.
\end{equation}
Mit \autoref{eq:tau} ergibt sich daraus
\begin{equation}
  \frac{W}{k_\text{B}T} = \ln{\frac{\int_{T}^{\infty} i(T^\prime) \text{d}T^\prime}{b \tau_0 \cdot i(T)}}\,,
  \label{eq:meth2}
\end{equation}
sodass $W$ aus der Steigung einer linearen Ausgleichsrechung bestimmt werden kann, indem die linke Seite dieser Gleichung gegen $1/T$ aufgetragen wird. Die obere Integrationsgrenze sollte zu diesem Zweck eine Temperatur sein, bei der der Strom Null ist. Dies gilt, nach Subtraktion des Untergrunds, sobald der Relaxationsprozess abgeschlossen ist.\par
Die Relaxationszeit $\tau_0$ kann bestimmt werden, indem \autoref{eq:stromdichte} nach der Temperatur abgeleitet wird. Das Ergebnis wird gleich Null gesetzt und an der Stelle $T_\text{max}$ ausgewertet:
\begin{equation}
  \tau_0 = \frac{k_\text{B} {T^2}_\text{max}}{W b} \cdot \exp{\left(- \frac{W}{k_\text{B} T_\text{max}}\right)}\,.
  \label{eq:tau0}
\end{equation}
$T_\text{max}$ ist die Temperatur, an der der Relaxationsstrom maximal wird.
