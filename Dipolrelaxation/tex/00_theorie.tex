\section{Theorie}

Durch Messung des Relaxationsstroms, der entsteht, wenn zuvor in einem elektrischen Feld ausgerichtete Dipole zurück in einen statistischen Zustand relaxieren, werden die Aktivierungsenergie und die Relaxationszeit von mit Strontium dotierten Kaliumbromid bestimmt.

\subsection{Permanente Dipole im Ionenkristall}

Durch das Dotieren des einwertigen Ionenkristalls Kaliumbromid mit zweiwertigen $\text{Strontium}^{2+}$-Ionen, entstehen Paare von Strontium-Ionen und Leerstellen, da eine lokale, elektrische Ladungsneutralität energetisch bevorzugt ist. So werden im Kristall permanante Dipole erzeugt. Die Richtungsverteilung der Dipole unterliegt der Boltzmannstatistik. Die Energie, die aufgebracht werden muss, um die Richtung eines Dipols im Gitter zu verändern beträgt $W$; die Anzahl der Dipole, die diese Energie bei einer Temperatur $T$ aufbringen können ist $\exp \left( -W/k_\text{B}T \right)$. Die Relaxationszeit $\tau$ ist eine Größe, die die mittlere Zeit zwischen zwei Umorientierungen eines Dipols beschreibt. Sie ist temperaturabhängig, nähert sich mit steigender Temperatur dem Wert $\tau_0$ an und lässt sich beschreiben mit
\begin{equation}
  \tau(T) = \tau_0 \exp \left( W/k_\text{B}T \right)\,.
\end{equation}

\subsection{Grundlagen des Messverfahrens}

Wird ein solcher Kristalls als Dielektrikum zwischen zwei Kondensatorplatten angebracht und ein elektrisches Feld angelegt, richtet sich ein Anteil der Dipole gemäß dem elektrischen Feld aus, sodass die Richtungsverteilung nicht mehr rein statistisch ist. Der Anteil der ausgerichteten Dipole ist dann näherungsweise
\begin{equation}
  y(p) = \frac{pE}{3k_\text{B}T}\,,
\end{equation}
wobei $p$ der Betrag des Dipolmoments und $E$ die elektrische Feldstärke ist.
