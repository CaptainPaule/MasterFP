\section{Diskussion}

Der Kontrast wurde zu $K = a \cdot | \sin (2 \cdot \phi + b) |$ mit $a = 1.06 \pm 0.05$. Das würde
bedeuten, dass bei einer optimalen Justage und dem optimalen Winkel am Polarisationsfilter von
$\SI{45}{\degree}$ mehr Licht an der Photodiode ankommt, als durch den Laser emittiert wird. Das kann
dadurch zustande kommen, dass eine externe Lichtquelle benutzt wurde, um den Versuchsaufbau zu beleuchten
und die Winkelskalen abzulesen und durch unterschiedliche Ausleuchtung verschieden viel Streulicht
auf die Photodioden gefallen ist. Ein solches Ergebnis ist in dem Fall nur realistisch, wenn die
Justage extrem genau war. Der maximale Kontrast wird sowohl in den Messergebnissen als auch im Fit bei
einem Winkel von $\SI{45}{\degree}$ erreicht.\\
\newline
Bei der Bestimmung des Brechungsindexes von Glas liefert die Ausgleichsrechung den Wert
$n_\text{Glas} = 1.553 \pm 0.009$. Fehler zum Theoriewert können nicht angegeben werden, da der
Brechungsindex von Glas zwischen $n = 1.4 \dots 1.8$ \cite{n} liegen kann. Damit besitzt der bestimmte
Wert eine realistische Größenordnung.\\
\newline
Der theoretische Wert für den Brechungsindex von Luft bei Normalbedingungen ist
$n_\text{Luft, theo} = 1.00027667$ \cite{n}, der gemessene Wert beträgt
$n_\text{Luft, exp} = 0.999547 \pm 0.000003$. Das bedeutet eine Abweichung vom Theoriewert um
$0.07\%$.
