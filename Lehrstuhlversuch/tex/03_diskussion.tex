\section{Diskussion}

Das drei Temperatur Modell stellt eine gute Näherung für das exponentielle Verhalten des Singals dar. Mit dem entsprechenden
Fit lässt sich der exponentielle Verhalten aus den Messungen für die Verschiedenen Gitter gut rausrechnen. 
In den korrigierten Signalen sind, zumindest in sinnvollen Grenzen, nurnoch die Oszillationen zu beobachten.
Außerdem ist zu beobachten, das die Amplitude der Schwingung von der Gittertiefe abhängt. Bei der Messung von flacheren Gittern z.b. 7nm is eine signifikante Abnahme im Vergleich zum Gitter mit 23nm
zu beobachten. 

\noindent Eine transformation in Frequenzdomäne zeigt einen Peak im Spektrum der Schwinung für ungefähr 12,5 GHz, was der charakteristichen Frequenz der ausbreitungsfähigen Schwinungen in dem untersuchten Gitter entspricht. Auch im Frequenzspektrum der einzelenen Proben lässt sich eine Korrelation von der 
Amplitude des Frequenzpeaks bei 12,5 GHz und der tiefe des dazugehörigen Gitter feststellen. Trägt man die Amplituden gegen die Gittertiefe auf ergibt sich ein näherungsweise linearer Zusammenhang. Die Messungen weiterer Proben mit unterschiedlichen Gittertiefen hätten dieses Ergebnis noch weiter konkretisieren können.

\noindent Abschließend ist noch anzumerken, das die Anwendung eines Tiefpass Filters mit einem Cut-Off von 16 GHz zu einer sichtbaren verbesserung der Signalqualität führt.

\printbibliography
