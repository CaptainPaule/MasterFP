\section{Theorie}

Mithilfe eines Anrege-Lasers werden in einem Festkörper akkustische Gitterschwingungen durch ultrakurze Laserpulse angeregt. Diese werden daraufhin mit einem Abfrage-Laser spektroskopisch untersucht.

\subsection{Good Vibrations}

In einem Galfenol-Film auf einem Galliumarsenid-Substrat werden akkustische Gitterschwingungen durch ultrakurze Laserpulse angeregt. Die Gitterschwingungen besitzen bei einem flachen Film ohne Oberflächenstruktur zunächst keine bevorzugte Frequenz. Die Intensität der Schwingung steigt also mit der Anregung durch den pump-Laser schlagartig an und fällt dann einem exponentiellen Abfall ähnlich ab.\par

Sobald die Oberflächenstruktur periodisch verändert wird, kommt es zu konstruktiver und destruktiver Interferenz, da es sich bei den Schwingungen um Rayleigh-Wellen handelt, welche an der Oberfläche lokalisiert sind. Rayleigh-Wellen haben eine Geschwindigkeitskomponente in Ausbreitungsrichtung und senkrecht dazu. Ihre Geschwindigkeit und Eindringtiefe sind abhängig von ihrer Frequenz.
