\section{Diskussion}
Die meisten Fehler sind unter 10\%, also unterhalb der Fehlertoleranz der elektronischen Bauteile. Fehlerquellen sind dennoch möglich, so besitzen die Ausgleichsrechnungen die angegebenen Fehler. Zum Anderen entstehen Abweichungen durch ohmsche Verluste und Ablesefehler. Bei den generierten Dreieck- und Rechtecksignalen sind zudem die Überschwinger, die aus dem Gibbschen Phänomen resultieren, zu erkennen. In \autoref{scope_247} scheint es bei der Oszilloskopaufnahme außerdem zu Reflexionen gekommen zu sein.\par
Bei der experimentellen Bestimmung der Triggerschwellen des Schmitt-Triggers sind vor allem die Ablesefehler ausschlaggebend. In \autoref{scope_237} sind zudem Knicke in der sinusförmigen Eingangsspannung zu erkennen, welche eine weitere Fehlerquelle darstellen können. Möglicherweise ist es hier zu einer unerwünschten Rückkopplung oder Interferenz durch das Ausgangssignal gekommen.\par
Die extrem große Abweichung zwischen theoretischer und experimentell bestimmter Frequenz bei dem Signalgenerator lässt sich nur durch Fehler bei der Durchführung erklären. Möglich ist zum Beispiel, dass andere Bauteile als dokumentiert verwendet wurden. Eine versehentliche Änderung der Betriebsspannung hätte keinen Effekt auf die Diskrepanz in der Frequenz zur Folge, würde aber eine Abweichung in der Amplitude erklären können.\par
Bei der Abklingdauer der gedämpften Schwingung ist eine weitere bedeutende Fehlerquelle, dass die Kapazitäten der Kondensatoren nicht exakt übereinstimmen. Auch hier spielt die Fehlertoleranz der elektronischen Bauteile eine große Rolle.

\printbibliography

\FloatBarrier

\section{Anhang}

\begin{table}
  \centering
\begin{tabular}{rrrr}
     $f$/Hz &  $U_A$/mV &  $\phi/\si{\degree}$ &  $U_1$/mV \\
\midrule
   0.01 &  960 &  173 &  198 \\
   0.10 &  960 &  173 &  198 \\
   1.00 &  960 &  173 &  198 \\
  10.00 &  950 &  168 &  198 \\
  25.00 &  920 &  163 &  198 \\
  50.00 &  900 &  153 &  200 \\
  75.00 &  850 &  145 &  202 \\
 100.00 &  810 &  139 &  205 \\
 125.00 &  760 &  131 &  207 \\
 200.00 &  610 &  114 &  217 \\
 250.00 &  530 &  107 &  220 \\
 275.00 &  500 &  102 &  221 \\
 300.00 &  470 &  100 &  220 \\
 325.00 &  430 &   96 &  218 \\
 350.00 &  410 &   93 &  219 \\
 400.00 &  370 &   90 &  225 \\
 450.00 &  340 &   87 &  221 \\
 500.00 &  310 &   82 &  225 \\
 550.00 &  290 &   80 &  225 \\
 600.00 &  270 &   75 &  225 \\
 650.00 &  250 &   72 &  225 \\
 700.00 &  235 &   70 &  225 \\
 750.00 &  230 &   70 &  225 \\
 800.00 &  220 &   65 &  225 \\
\end{tabular}
\caption{Messwerte für den Frequenzgang eines gegengekoppelten Verstärkers mit $R_1 = \SI{0.2}{\kilo\ohm}$ und $R_\text{N} = \SI{1}{\kilo\ohm}$.}
\end{table}

\begin{table}
  \centering
\begin{tabular}{rrrr}
      $f$/Hz &  $U_A$/mV &  $\phi/\si{\degree}$ &  $U_1$/mV \\
\midrule
   0.010 &  736 &  173 &  235 \\
   0.025 &  736 &  172 &  235 \\
   0.050 &  736 &  172 &  235 \\
   0.100 &  736 &  172 &  235 \\
   0.500 &  736 &  172 &  235 \\
   1.000 &  736 &  171 &  235 \\
  10.000 &  736 &  170 &  235 \\
  50.000 &  716 &  160 &  235 \\
  75.000 &  695 &  155 &  235 \\
 100.000 &  679 &  150 &  235 \\
 125.000 &  553 &  143 &  235 \\
 250.000 &  511 &  118 &  235 \\
 375.000 &  400 &   98 &  235 \\
 400.000 &  380 &   96 &  235 \\
 425.000 &  362 &   93 &  235 \\
 450.000 &  346 &   90 &  235 \\
 475.000 &  328 &   88 &  235 \\
 500.000 &  316 &   85 &  235 \\
 525.000 &  302 &   83 &  235 \\
 550.000 &  291 &   81 &  235 \\
 575.000 &  279 &   79 &  235 \\
 600.000 &  267 &   77 &  235 \\
 625.000 &  259 &   74 &  235 \\
 650.000 &  251 &   73 &  235 \\
 675.000 &  239 &   70 &  235 \\
 700.000 &  231 &   69 &  235 \\
\end{tabular}
\caption{Messwerte für den Frequenzgang eines gegengekoppelten Verstärkers mit $R_1 = \SI{10.02}{\kilo\ohm}$ und $R_\text{N} = \SI{33.2}{\kilo\ohm}$.}
\end{table}

\begin{table}
  \centering
\begin{tabular}{rrrr}
     $f$/Hz &  $U_A$/mV &  $\phi/\si{\degree}$ &  $U_1$/mV \\
\midrule
   0.01 &  2.49 &  172 &  0.233 \\
   0.10 &  2.37 &  172 &  0.233 \\
   0.50 &  2.37 &  172 &  0.233 \\
   1.00 &  2.31 &  171 &  0.233 \\
  10.00 &  2.20 &  164 &  0.233 \\
  25.00 &  2.10 &  155 &  0.233 \\
  50.00 &  1.90 &  141 &  0.233 \\
  75.00 &  1.66 &  129 &  0.233 \\
 100.00 &  1.44 &  119 &  0.233 \\
 125.00 &  1.24 &  112 &  0.233 \\
 150.00 &  1.08 &  106 &  0.233 \\
 175.00 &  0.95 &  100 &  0.233 \\
 200.00 &  0.85 &   96 &  0.233 \\
 225.00 &  0.77 &   92 &  0.233 \\
 250.00 &  0.71 &   90 &  0.233 \\
 275.00 &  0.65 &   85 &  0.233 \\
 300.00 &  0.60 &   82 &  0.233 \\
 325.00 &  0.55 &   80 &  0.233 \\
 350.00 &  0.53 &   79 &  0.233 \\
 375.00 &  0.49 &   78 &  0.233 \\
 450.00 &  0.43 &   70 &  0.233 \\
 400.00 &  0.47 &   75 &  0.233 \\
 500.00 &  0.39 &   65 &  0.233 \\
 550.00 &  0.37 &   65 &  0.233 \\
 600.00 &  0.33 &   60 &  0.233 \\
 650.00 &  0.31 &   55 &  0.233 \\
 700.00 &  0.29 &   50 &  0.233 \\
 750.00 &  0.27 &   50 &  0.233 \\
 800.00 &  0.25 &   45 &  0.233 \\
 850.00 &  0.23 &   45 &  0.233 \\
 900.00 &  0.23 &   45 &  0.233 \\
\end{tabular}
\caption{Messwerte für den Frequenzgang eines gegengekoppelten Verstärkers mit $R_1 = \SI{10.02}{\kilo\ohm}$ und $R_\text{N} = \SI{100}{\kilo\ohm}$.}
\end{table}

\begin{table}
  \centering
\begin{tabular}{rrrr}
     $f$/Hz &  $U_A$/mV &  $\phi/\si{\degree}$ &  $U_1$/mV \\
\midrule
   0.01 &    NaN &  750 &  175 \\
   0.10 &  720.0 &  175 &  231 \\
   0.25 &    NaN &  710 &  173 \\
   0.50 &  710.0 &  173 &  231 \\
   1.00 &  710.0 &  173 &  231 \\
  10.00 &  700.0 &  170 &  231 \\
  25.00 &  680.0 &  167 &  231 \\
  50.00 &  680.0 &  160 &  231 \\
  75.00 &  670.0 &  155 &  231 \\
 100.00 &  660.0 &  150 &  231 \\
 175.00 &  610.0 &  130 &  231 \\
 250.00 &  540.0 &  112 &  231 \\
 300.00 &  480.0 &  100 &  231 \\
 375.00 &  410.0 &   86 &  231 \\
 450.00 &  336.0 &   74 &  231 \\
 500.00 &  304.0 &   66 &  231 \\
 525.00 &  287.0 &   61 &  231 \\
 550.00 &  269.0 &   59 &  231 \\
 575.00 &  254.0 &   56 &  231 \\
 600.00 &  243.0 &   52 &  231 \\
 625.00 &  231.0 &   50 &  231 \\
\end{tabular}
\caption{Messwerte für den Frequenzgang eines gegengekoppelten Verstärkers mit $R_1 = \SI{33.3}{\kilo\ohm}$ und $R_\text{N} = \SI{100}{\kilo\ohm}$.}
\end{table}

\begin{table}
  \centering
\begin{tabular}{rrrr}
    $f$/Hz &  $U_E$/mV &   $U_A$/V &  $\phi/\si{\degree}$ \\
\midrule
 0.001 &  235 &  27.1 &  172.6 \\
 0.010 &  235 &  27.1 &  160.5 \\
 0.100 &  235 &  27.1 &  122.2 \\
 0.200 &  192 &  27.1 &   96.1 \\
 0.300 &  192 &  19.5 &   94.2 \\
 0.400 &  199 &  14.7 &   95.0 \\
 0.500 &  192 &  11.7 &   95.0 \\
 0.600 &  199 &  10.3 &   94.3 \\
 0.700 &  192 &   8.8 &   94.0 \\
 0.800 &  199 &   7.6 &   95.0 \\
 0.900 &  199 &   6.9 &   96.0 \\
 1.000 &  198 &   6.2 &   96.0 \\
 1.100 &  199 &   5.7 &   96.0 \\
 1.200 &  198 &   5.3 &   95.0 \\
 1.300 &  199 &   4.9 &   94.0 \\
 1.400 &  199 &   4.6 &   96.0 \\
 1.500 &  189 &   4.4 &   98.0 \\
 1.600 &  199 &   4.1 &   99.0 \\
\end{tabular}
\caption{Messwerte für den Umkehr-Differentiator mit $R_1 = \SI{10.02}{\kilo\ohm}$ und $C_1 = \SI{23.1}{\nano\farad}$.}
\label{tab:d}
\end{table}

\begin{table}
  \centering
\begin{tabular}{rrrr}
  $f$/Hz &  $U_E$/mV &   $U_A$/V &  $\phi/\si{\degree}$ \\
\midrule
 1.0 &  310 &  13.0 & -88.0 \\
 1.1 &  310 &  13.8 & -91.0 \\
 1.2 &  310 &  14.7 & -88.0 \\
 1.3 &  310 &  15.7 & -90.0 \\
 1.4 &  310 &  16.3 & -85.9 \\
 1.5 &  308 &  17.5 & -88.1 \\
 1.7 &  310 &  18.9 & -92.1 \\
 1.9 &  310 &  20.5 & -91.1 \\
 2.1 &  310 &  22.5 & -93.0 \\
 2.3 &  310 &  24.1 & -92.2 \\
 2.5 &  320 &  25.7 & -94.2 \\
 2.7 &  320 &  27.7 & -91.0 \\
\end{tabular}
\caption{Messwerte für den Umkehr-Integrator mit $R_1 = \SI{0.2}{\kilo\ohm}$ und $C_1 = \SI{23.1}{\nano\farad}$.}
\end{table}

