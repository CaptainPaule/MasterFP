\section{Diskussion}
Alle Fehler sind unter 10\%, also unterhalb der Fehlertoleranz der elektronischen Bauteile. Fehlerquellen sind dennoch möglich, so besitzen die Ausgleichsrechnungen die angegebenen Fehler. Zum Anderen entstehen Abweichungen durch ohmsche Verluste und Ablesefehler. Bei den generierten Dreieck- und Rechtecksignalen sind zudem die Überschwinger, die aus dem Gibbschen Phänomen resultieren, zu erkennen. In \autoref{scope_247} scheint es bei der Oszilloskopaufnahme außerdem zu Reflexionen gekommen zu sein.\par
Bei der Abklingdauer der gedämpften Schwingung ist eine weitere bedeutende Fehlerquelle, dass die Kapazitäten der Kondensatoren nicht exakt übereinstimmen.

\printbibliography
