\section{Durchf{\"u}hrung}
Zuerst wird eine Schaltung nach \autoref{abb:linear} aufgebaut und für vier verschiedene Verstärkungen (einstellbar über den Widerstand $R_N$ in der Gegenkopplung oder den Eingangswiderstand $R_1$)
die Frequenz über mehrere Größenordnungen variert. Es ist eine frequenzabhängige Abnahme der Verstärkung zu beobachten und eine Abhängigkeit der Phase von der Frequenz wird überprüft.

Als nächstes wird ein Umkehrintegrator nach \autoref{abb:int} aufgebaut, sodass die
Frequenzabhängigkeit der Amplitude des Ausgangssignals untersucht werden kann. Es
ist darauf zu achten, dass der Frequenzbereich so gewählt wird, das der Umkehrintegrator
auch als solcher arbeitet. Dieses Vorgehen wiederholt man für den Differentiator nach \autoref{abb:diff}.

Als nächste Schaltung wird der Schmitt-Triger nach \autoref{abb:sch} untersucht.
An den Eingang des Schmitt-Triggers wird ein Funktionsgenrator angeschlosssen und an den Ausgang
ein Oszilloskop. Die Amplitude des Eingangssignals wird von null ausgehend solange erhöht,
bis die Schaltung anfängt zu kippen. Daraufhin wird der Scheitelwert dieser Spannung sowie
sie Größe $2U_B$ gemessen.

Für den Dreiecksgenerator nach \autoref{abb:sig} wird die Zeitabhängigkeit der Ausgangspannung mit einem
Oszilloskop kontrolliert sowie die Frequenz und die Amplitude des erzeugten Signals gemessen.

Abschließend wird ein Signalgenerator nach \autoref{abb:sig2} aufgebaut. An den Ausgang der Schaltung
wird ein Oszilloskop angeschlossen. Über das Potentiometer ist es möglich die Dämpfung der Schwingungs
einzustellen. Das Potentiometer wird so eingestellt, dass das Ausgangssignal ungedämpft ist. Von diesem Signal wird
die Frequenz bestimmt. Daraufhin wird die Dämpfung auf ein Maximum erhöht und
die Schwingung mit einem Rechteck-Signal angeregt, wobei die Periodendauer des Eingangssignals groß ist
gegenüber der Abklingdauer der gedämpften Schwingung. Von dem Ausgangssignal wird ein Bild erstellt.
