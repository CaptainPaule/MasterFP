\section{Theorie}

\subsection{Ideale und reale Operationsverst{\"a}rker}
Ein Operationsverstärker ist ein gleichstromgekoppelter Differenzverstärker. Die Spannung am Ausgang $U_A$ entspricht der Differenz der Eingangsspannungen $U_P$ (am nicht invertierenden Eingang)
sowie $U_N$ (am invertierenden Eingang), die um den Verstärkungsfaktor $V$ verstärkt wird.
\begin{equation}
U_A = V (U_P - U_N)
\end{equation}
Das Schaltbild mit den entsprechenden Anschlüssen ist in \autoref{abb:op} abgebildet.
\begin{figure}[h!]
 	\centering
 	\includegraphics[width=\textwidth]{img/op.png}
 	\caption{Schaltsymbol des Operationsverstärkers \cite{FP}}
 	\label{abb:op}
\end{figure}
Die Ausgangsspannung des Operationsverstärkers kann im Bereich der Betriebsspannungen varrieren:
\begin{equation}
-U_B < U_A < U_B
\end{equation}
Sollte $U_A$ außerhalb des von den Betriebsspannungen definierten Bereich leigen, so nimmt die Ausgangspannung den nächstliegenden Wert, also $-U_B$ oder $U_B$ an.

Um den Umgang mit realen Operationsverstärkern bei Berechnungen und in Schaltungen zu vereinfachen, wird das Modell des idealen Operationsverstärkers eingeführt:
Der ideale Operationsverstärker zeichnet sich durch eine unendlich große Leerlaufverstärkung, einem unendlichen Eigangswiderstand, sowie einem verschwindenen Eingangswiderstand aus.

Für reale Operationsverstärker gibt es weitere Eigenschaften, die erläutert werden müssen:
Sollte an beiden Eingängen die gleiche Spannung $U_G$ anliegen, müsste theoretisch $U_A = 0$ sein. Jedoch beobachtet man auf Grund der Unsymmetrien der beiden Kanäle eine Ausgangsspannung.
Der Quotient aus $U_G$ und $U_A$ wird Gleichtacktverstärkung
\begin{equation}
V_G = \frac{\Delta U_A}{\Delta U_G}
\end{equation}
genannt und ist ein Maß für die Abweichung vom idealen Operationsverstärker.
Ferner gibt es Eingangsströme auf Grund der endlichen Eingangswiderstände. Somit ist es möglich den Eingangsruhestrom
\begin{equation}
I_B = \frac{1}{2} (I_P + I_N)
\end{equation}
aus den Eingangsströmen $I_P$ (nicht invertierter Eingang) und $I_N$ (invertierter Eingang) zu berechnen.

\noindent Außerdem lässt sich der Offsetstrom definieren
\begin{equation}
I_0 = I_P - I_N.
\end{equation}
Aus den definierten Größen lässt sich der Differenzeingangswiderstand $r_D$ definieren:
\begin{equation}
r_D = \left\{\begin{array}{ll} \frac{\Delta U_P}{\Delta I_P}, \, \, U_N = 0\\
	  \frac{\Delta U_N}{\Delta I_N}, \, \, U_P = 0\end{array}\right\}.
\end{equation}
Auch ist für $U_G = U_P = U_N$ und $I_G = I_P + I_N$ der Gleichtakteingangswiderstand
\begin{equation}
r_G = \frac{\Delta U_G}{\Delta I_G}
\end{equation}

\subsection{Schaltungen mit Operationsverst{\"a}rker}

\subsubsection{Linearverst{\"a}rker}
Da der Operationsverstärker eine große Leerlaufverstärkung besitzt, kann  er nur einen kleinen Eingangsspannungsbereich linear verstärken, bevor er an die Grenzen  seiner Betriebsspannung stößt und in Sättigung geht. Um den Aussteuerungsbereich zu vergrößern, wird der Operationsverstärker nach \autoref{abb:linear} mit einer Gegegenkopplung erweitert, mit der das Verstärkungsverhältnis eingestellt werden kann.

\begin{figure}[h!]
	\centering
	\includegraphics[width=\textwidth]{img/linear.png}
	\caption{Gegengekoppelter invertierender Linearverstärker \cite{FP}}
	\label{abb:linear}
\end{figure}

\noindent Die Verstärkung des Linearverstärkers hat somit die Form
\begin{equation}
V^\prime = - \frac{R_N}{R_1}
\end{equation}
bzw. unter Berücksichtigung der Eigenschaften eines realen Operationsverstärkers
\begin{equation}
\frac{1}{V^\prime} = - \frac{U_1}{U_A} = \frac{1}{V} + \frac{R_1}{R_N} \left( 1 + \frac{1}{V} \right) \approx \frac{1}{V} + \frac{R_1}{R_N}.
\label{effVerstaerkung}
\end{equation}


\subsubsection{Elektrometerverst{\"a}rker}
Bei Messungen mit hochohmigen Spannungsquellen ist es möglich, dass der geringe Eingangswiderstand des Linearverstärkers die Messung Verfäschen. Die
Elektrometerverstärkerschaltung nach \autoref{abb:elektro} besitzt diesen Nachteil nicht, da die Eingangsspannung direkt am invertierenden Eingang des Operationsverstärkers anliegt, was für einen hohen Eingangswiderstand
sorgt.
\begin{figure}[h!]
 	\centering
 	\includegraphics[width=\textwidth]{img/elec.png}
 	\caption{nicht-invertierender Elektrometerverstärker \cite{FP}}
 	\label{abb:elektro}
\end{figure}
Hier berechet sich der Verstärkungsfaktor gemäß:
\begin{equation}
V' = \frac{U_A}{U_1}=\frac{R_N + R_1}{R_1}.
\end{equation}


\subsubsection{Amperemeter}
Zur Messung von kleinen Strömen wird ein kleiner Eingangswiderstand benötigt. Dazu kann ein Amperemeter nach \autoref{abb:ampere} verwendet werden, bei dem die Ausgangsspannung proportional zum Eingangsstrom ist:
\begin{equation}
U_A = IR_N.
\end{equation}
\begin{figure}[h!]
 	\centering
 	\includegraphics[width=\textwidth]{img/ampere.png}
 	\caption{Amperemeter mit niedriegen Eingangswiderstand \cite{FP}}
 	\label{abb:ampere}
\end{figure}
Der Eingangswiderstand der Schaltung ist
\begin{equation}
r_e = \frac{R_N}{V}
\end{equation}

\subsubsection{Integrator und Differentiator}
Mit der in \autoref{abb:int} dargestellten Schaltung wird die Eingangsspannung integiert:
\begin{figure}[h!]
 	\centering
 	\includegraphics[width=\textwidth]{img/int.png}
 	\caption{Umkehr-Integrator \cite{FP}}
 	\label{abb:int}
\end{figure}
\begin{equation}
U_A = - \frac{1}{RC} \int U_1(t) \text{dt} .
\end{equation}
Beschreibt $U_1$ eine Sinusspannung $U_1 = U_0 \sin(\omega t)$, so ist die Amplitude antiproportional zur Frequenz
\begin{equation}
U_A = \frac{U_0}{\omega RC} \cos(\omega t) .
\end{equation}

\noindent Das Gegenstück zum Integrator ist der Differentiator. Dieser wird nach \autoref{abb:diff} aufgebaut.
\begin{equation}
U_A = -RC \frac{\text{d} U_1}{\text{d} t} .
\end{equation}
Für den Fall, dass das Eingangssignal wieder als Sinusspannung vorliegt ergibt sich:
\begin{equation}
U_A = -\omega RCU_0 \cos(\omega t) .
\end{equation}
Somit ist die Amplitude der Ausgangsspannung proportional zur Frequenz

\begin{figure}[h!]
 	\centering
 	\includegraphics[width=\textwidth]{img/diff.png}
 	\caption{Umkehr-Differentiator \cite{FP}}
 	\label{abb:diff}
\end{figure}


\subsubsection{Schmitt-Trigger}
Der Schmitt-Trigger wird nach \autoref{abb:sch} verschaltet. Hier liegt ein Teil der Ausgangsspannung wieder am  nicht-invertierten Eingang an. Dadurch steigt die Ausgangsspannung immer weiter an. Die Schaltung bekommt somit ein instabiles Verhalten. Der Schmitt-Triger hat nur zwei Zusände, in denen seine Ausgangsspannung $U_B$ beträgt, wenn die Eingangsspannung $(R_1 / R_P) \cdot U_B$ überschreitet oder
$-U_B$ wenn die Eingangsspannung $-(R_1 / R_P) \cdot U_B$ unterschreitet. Somit ist der Schmitt-Trigger als Schalter zu verstehen, welcher nicht nur einen Schwellenwert besitzt.  Durch geeignete Wahl der Widerstände ist es möglich, das Hysterese-Fenster zu beeinflussen und somit Einfluss auf die Schwellenwerte der Schaltung zu nehmen Der Schmitt-Trigger ist daher ein nützliches Element für binäre Logik und Signalgeneratoren.

\begin{figure}[h!]
 	\centering
 	\includegraphics[width=\textwidth]{img/schmitt.png}
 	\caption{Schmitt-Trigger \cite{FP}}
 	\label{abb:sch}
\end{figure}

\subsubsection{Signalgenerator}
In \autoref{abb:sig} ist der Aufbau eines Signalgenerators gezeigt. Dieser besteht aus einem Schmitt-Trigger und einem Integrator. Der Schmitt-Trigger liefert zu beginn eine kostante Ausgangsspannung $U_B$,
welche vom Integrator integriert wird. Die Ausgangsspannung der Integrator fällt somit linear mit der Zeit bis diese die Triggerschwelle des Schmitt Triggers erreicht hat, sodass der Schmitt-Trigger Schaltet und eine konstante Ausgangsspannung $-U_B$ liefert. Diese wird vom Integrator wieder Integriert was dazu führt das seine Ausgangspannung mit der Zeit linear ansteigt, bis die andere Triggerschwelle erreicht wird. Sodann schaltet der Schmitt-Trigger erneut und ist wieder im Ausgangszustand.

Die Integration erfolgt folgendermaßen:
\begin{equation}
U_A = - \frac{1}{RC} \int_{0}^{\frac{T}{2}} U_E(t^\prime)dt^\prime = -\frac{1}{RC} U_B \frac{T}{2} .
\end{equation}
Da diese Spannung der Differenz zwischen den beiden Schwellenwerten des Schmitt-Triggers entspricht, ist
\begin{equation}
U_A = 2U_B\frac{R_1}{R_P} .
\end{equation}
Diese beiden Gleichung ergeben mit der Periodendauer $T$ und der Frequenz $f$:
\begin{equation}
f = \frac{R_P}{4RCR_1} .
\end{equation}

\begin{figure}[h!]
 	\centering
 	\includegraphics[width=\textwidth]{img/gen.png}
 	\caption{Dreiecks- und Rechteckgenerator \cite{FP}}
 	\label{abb:sig}
\end{figure}

\subsubsection{Erzeugung von ged{\"a}mpften Sinusschwingungen}
Es wird ein Generator verwendet, der eine Sinusschwingung erzeugt, die mit einer abfallenden Exponentialfunktion überlagert ist. Die Schaltung ist in \autoref{abb:sig2} dargestellt.

\noindent Die Schaltung besitzt die Differentialgleichung
\begin{equation}
\frac{\text{d}^2 U_A}{\text{d}t^2} - \frac{\nu}{10RC} \frac{\text{d} U_A}{\text{d}t} + \frac{1}{R^2 C^2} U_A = 0
\end{equation}
mit der näherungsweise Lösung
\begin{equation}
U_A(t) = U_0 \exp\left(\frac{\nu t}{20 RC}\right) \sin\left(\frac{t}{RC}\right) .
\end{equation}
Mit der Schwingungsdauer
\begin{equation}
T = 2\pi RC
\end{equation}
und der Abklingdauer
\begin{equation}
\tau = \frac{20 RC}{\nu} .
\end{equation}

\begin{figure}[h!]
 	\centering
 	\includegraphics[width=\textwidth]{img/sin.png}
 	\caption{Nachbildung einer linearen Schwingungsdifferentialgleichung mit Operationsverstärkern \cite{FP}}
 	\label{abb:sig2}
\end{figure}
