\section{Diskussion}
Die errechneten g-Faktoren für Rb85 und Rb87 mit 0,488 $\pm$ 0,005 und
0,331 $\pm$ 0,002 stimmen mit guter Genauigkeit mit den Theoriewerten von $\frac{1}{2}$ und $\frac{1}{3}$ überein.
Auch der Wert für die horizontal Komponente des Erdmagnetfeldes scheint unter Berücksichtigung der Größenordnung plausiebel zu sein.

Der Vergleich der Kernspins mit den Literaturwerten liefert ebenso eine zufriedenstellende Übereinstimmung.
Für Rb85 finden wir einen Kernspin von  I = $\frac{5}{2}$ und für Rb87 einen Kernspin von I = $\frac{3}{2}$, was im wesentlichen den
Literaturwerten entspricht.

Das errechnete Isotopenverhältnis von r = 0,507 weicht signifikant vom Literaturwert von r = 0,386 ab. Als mögliche Fehlerquelle
sei hier eine möglicherweise nicht komplett abgedunkelte Apparatur genannt. In diesem Fall hätte das einfallende Restlicht
einen Offset in der Amplitudentiefe der Resonanzen bewirkt.

Auch könnte die Speicherung als JPEG und die weitere Verarbeitung mit gimp einen Fehler begründen.

Die Abschätzung des quadratischen Zeemann-Effektes zeigt, dass der Einfluss dieses Effektes bei den verwendeten Magnetfeldern in der
Größenordnung $\approx 10^{-33} \text{J}$ liegt. Demnach können wir den Effekt, zumindest bei Feldern mit $\mu$T Bereich vernachlässigen.
