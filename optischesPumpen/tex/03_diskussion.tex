\section{Diskussion}
Die errechneten g-Faktoren für Rb85 und Rb87 mit $g_{F,85} = 0,488 $\pm$ 0,005$ und
$g_{F,87} = 0,331 $\pm$ 0,002$ stimmen mit guter Genauigkeit mit den Theoriewerten von $\frac{1}{2}$ und $\frac{1}{3}$ überein.
Der errechnete Wert für die horizontal Komponente des Erdmagnetfeldes $B_{h} = 20.3\pm0.5\,\mu\text{T}$ weicht um 5 Prozentpunkte vom
Literaturwert $B_{h,Theorie} = 19,3\mu\text{T}$ \cite{hB} ab.

Der Vergleich der Kernspins mit den Literaturwerten liefert ebenso eine zufriedenstellende Übereinstimmung.
Für Rb85 finden wir einen Kernspin von  $I_\text{Rb,85} = \frac{5}{2}$ und für Rb87 einen Kernspin von $I_\text{Rb,87} = \frac{3}{2}$, was im wesentlichen den Literaturwerten \cite{coreSpin} entspricht.

Das errechnete Isotopenverhältnis von r = 0,507 weicht signifikant vom Literaturwert von r = 0,386 \cite{isoVer} ab. Als mögliche Fehlerquelle sei hier eine möglicherweise nicht komplett abgedunkelte Apparatur genannt. In diesem Fall hätte das einfallende Restlicht
einen Offset in der Amplitudentiefe der Resonanzen bewirkt.

Auch könnte die Speicherung als JPEG und die weitere Verarbeitung mit gimp einen Fehler begründen.

Die Abschätzung des quadratischen Zeemann-Effektes zeigt, das der quadratische Term bei den verwendeten Feldstärken drei Größenordnungen kleiner ist als der lineare Term. Daher kann der Einfluss des quadratischen Terms in guter Näherung vernachlässigt werden.

Der Theoriewerte des Verhältnisses $\frac{b_{87}}{b_{85}}$ ist 1,5 \cite{FP}. Somit weicht das Messergebnis von $\frac{b_{87}}{b_{85}} = 1,17 \pm 0,06$ um 22 Prozenpunkte vom Theoriewert ab.
Dieser Abweichung lässt sich damit begründen, dass das Zählen der Perioden nicht mit absoluter Genauigkeit möglich war.

\printbibliography
