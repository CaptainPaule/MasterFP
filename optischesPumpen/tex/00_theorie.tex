\section{Zielsetzung}
Durch Messung der Transmission von Licht durch Rubidiumgas bei Anlegen eines externen Magnetfeldes werden die
Stärke des Erdmagnetfeldes, die Lande-Faktoren und die Spins der Elektronenhülle und des Kerns der Rubidiumisotope
Rb-85 und Rb-87 berechnet.
Für die beiden verwendeten Rubidiumisotope sind Hyperfeinstruktur- und Zeeman-Aufspaltung unterschiedlich. Die damit
verbundenen Kenngrößen - der Landefaktor, das gyromagnetische Verhältnis und der Spin - können mit Hilfe optischen
Pumpens aufgrund der unterschiedlichen Enerdifferenzen sehr präzise gemessen werden.

\section{Theorie}

\subsection{Aufspaltung von Energieniveaus}

Quantenzahlen beschreiben Zustand eines atomareb Systems.
Das magnetische Moment der gesamten Elektronenhülle $\vec{J}$ setzt sich aus den magnetischen Momenten von
Bahndrehimpuls $\vec{L}$ und Spin $\vec{S}$ zusammen:
\begin{equation}
  \vec{\mu}_\text{J} = \vec{\mu}_\text{L} + \vec{\mu}_\text{S} \quad \text{mit} \quad |\vec{\mu}_\text{J}| = \mu_\text{B}\sqrt{L (L + 1)} \quad \text{und}
  \quad |\vec{\mu}_\text{S}| = g_\text{S}\mu_\text{B}\sqrt{S (S + 1)}.
\end{equation}
Der Gesamtdrehimpuls der Elektronenhülle $\vec{J}$ ist über
\begin{equation}
  \vec{\mu}_\text{J} = -g_\text{J}\mu_\text{B}\vec{J} \quad \text{bzw.} \quad |\vec{\mu}_\text{J}| = -g_\text{J}\mu_\text{B}\sqrt{J (J + 1)}
  \label{magnMom}
\end{equation}
mit einem magnetischen Moment verknüpft, dabei ist $\mu_\text{B}$ das Bohrsche Magneton.
$g_\text{S}$ ist der Lande-Faktor des freien Elektrons, der Zusammenhang der magnetischen Momente wird durch den
Lande-Faktor $g_\text{J}$ ausgedrückt.
%$\vec{\mu}_J$ präzediert um $\vec{J}$, sodass für den Betrag gilt:
%\begin{align}
%  \begin{split}
%    |\vec{\mu}_J| &= |\vec{\mu}_L|\cos\beta + |\vec{\mu}_S|\cos\alpha\\
%    \equiv g_J\mu_B\sqrt{J (J + 1)} = & \mu_B\sqrt{L (L + 1)}\cos\beta + g_S\mu_B\sqrt{S (S + 1)}\cos\alpha
%  \end{split}
%\end{align}
Dieser kann aus den Quantenzahlen über
\begin{equation}
  g_\text{J} = \frac{3.0023J(J+1) + 1.0023[S(S+1) - L(L+1)]}{2J(J+1)}
\end{equation}
berechnet werden.

Der sogenannte Zeeman-Effekt beschreibt die Aufspaltung in $2J+1$ Energieniveaus bei Anlegen eines äußeren
Magnetfeldes. Liegt ein äußeres Magnetfeld $\vec{B}$ an, kommt es zu einer Aufspaltung der durch $L$, $S$ und
$J$ definierten Feinstruktur-Energieniveaus, da das magnetische Moment $\vec{\mu}$ mit dem Feld wechselwirkt; die
Wechselwirkungsenergie ist
\begin{equation}
  U_\text{magn} = -\vec{\mu}_\text{J}\cdot\vec{B}.
\end{equation}
Da nur die zu $\vec{B}$ parallele Komponente des magnetischen Moments zu diesem Effekt beiträgt, kommt es zu einer Aufspaltung
der Energieniveaus gemäß
\begin{equation}
  U_\text{magn} = M_\text{J}g_\text{J}\mu_\text{B}B,
\end{equation}
wobei $M_\text{J}$ ganzzahlig ist und aus $-J, -J+1, ..., J-1, J$ stammt.

Bei einem nicht-verschwindenden Kernspin $\vec{I}$ kommt es zudem zu einer Aufspaltung in Hyperfeinstrukturniveaus.
Der Gesamtdrehimpuls der Elektronenhülle $\vec{J}$ koppelt an den Drehimpuls des Kerns $\vec{I}$ zum Gesamtdrehimpuls des Atoms
\begin{equation}
  \vec{F} = \vec{J} + \vec{I}.
\end{equation}
Die Energiedifferenz benachbarter Zeeman-Niveaus ist dann
\begin{equation}
  U_\text{UF} = g_\text{F}\mu_\text{B}B.
\end{equation}
Der Lande-Faktor ist in diesem Fall
\begin{equation}
  g_\text{F} \approx g_\text{J}\frac{F(F+1)+J(J+1)-I(I+1)}{2F(F+1)}.
\end{equation}
$F$ läuft von $I+J$ bis $|I-J|$, jedes Hyperfeinniveau spaltet in einem äußeren Magnetfeld in weiter $2F+1$ Zeeman-Niveaus
auf. Für ein Alkali-Atom mit $I=\frac{3}{2}$ ist dies in \autoref{aufspaltung} beispielhaft gezeigt.
\begin{figure}
  \centering
  \includegraphics[width=\textwidth]{img/aufspaltung.pdf}
  \caption{Feinstruktur-, Hyperfeinstruktur- und Zeeman-Aufspaltung, beispielhaft für ein Alkali-Atom mit
  $I=\frac{3}{2}$ \cite{FP}.}
  \label{aufspaltung}
\end{figure}

\subsection{Optisches Pumpen}

Die Energieniveaus der inneren Schalen der Elektronenhülle sind vollstandig besetzt, die Besetzung der äußeren Schalen ist
temperaturabhängig. Im thermischen Gleichgewicht wird sie für zwei Niveaus mit den Energien $W_1$ und $W_2$ durch die
Boltzmannsche Gleichung
\begin{align}
  \frac{N_2}{N_1} = \frac{g_2}{g_1}\frac{\exp(-W_2/kT)}{\exp(-W_1/kT)}
  \label{boltz}
\end{align}
beschrieben. $N_i$ sind die Besetzungszahlen der jeweiligen Zustände und die statistischen
Gewichte $g_i$ sind ein Maß dafür, wie viele Zustände es pro Energie $W_i$ gibt.

Der Begriff Optisches Pumpen bezeichnet eine Methode, mit der eine Abweichung von der in
\autoref{boltz} gegebenen Verteilung erzielt werden kann - zum Beispiel in Form einer Inversion,
bei der $N_2 > N_1$ gilt. Dafür wird Licht eingestrahlt, das gerade die nötige Energie besitzt,
um ein Hüllenelektron vom Grundzustand in einen angeregten Zustand zu versetzen. Für ein
Alkali-Atom mit $J=\frac{1}{2}$ beispielsweise kann $M_\text{J}$ also nur $\pm \frac{1}{2}$
werden. Da bei der Absorption und Emission eines Photons Drehimpulserhaltung gelten muss, sind
nur bestimmte Übergänge zwischen unterschiedlichen Energieniveaus möglich; diese Übergänge sind
über Auswahlregeln definiert. Aufgrund dieser Auswahlregeln sind nur Übergänge mit
$\Delta M = 0,\pm 1$ möglich.
Bei $\pi$-Übergangen mit $\Delta M = 0$ wird linear polarisiertes Licht emitiert und absorbiert, für $\sigma^\pm$-Übergänge
mit $\Delta M = \pm 1$ ist es rechtszirkular- bzw. linkszirkular-polarisiertes Licht.

Ein Alkali-Atom mit einem Hüllenelektron, bei dem der Kernspin vernachlässigt wird, besitzt
den Grundzustand $\isotope[2]{S}_{\sfrac{1}{2}}$, also insgesamt die Quantenzahlen
$S=\sfrac{1}{2}$, $L=0$ und $J=\sfrac{1}{2}$. Die ersten angeregten Zustände sind
$\isotope[2]{P}_{\sfrac{1}{2}}$ ($S=\sfrac{1}{2}$, $L=1$, $J=\sfrac{1}{2}$) und
$\isotope[2]{P}_{\sfrac{3}{2}}$ ($S=\sfrac{1}{2}$, $L=1$ und $J=\sfrac{3}{2}$). Diese drei
Zustände bilden das $D_1D_2$-Dublett, im Folgenden wird nur der $D_1$-Übergang nach
$\isotope[2]{P}_{\sfrac{1}{2}}$ betrachtet.
Bei eingeschalteten Magnetfeld $\vec{B}$ findet eine Zeeman-Aufspaltung in Niveaus mit
$M_\text{J}=\pm\sfrac{1}{2}$ statt, die Zustände sind in \autoref{uebergang} zu sehen.

\begin{figure}
  \centering
  \includegraphics[width=\textwidth]{img/uebergang.png}
  \caption{Zustände des $D_1$-Überganges bei einem Alkali-Atom mit $S=\sfrac{1}{2}$ und
  vernachlässigtem Kernspin \cite{FP}.}
  \label{uebergang}
\end{figure}

Wird rechtszirkular-polarisiertes $D_1$-Licht in eine Zelle mit verdampftem Rubidium eingestrahlt,
können aufgrund der Beschränkung $\Delta M_\text{J} = 1$ nur Elektronen aus dem energetisch
niedrigeren Grundzustand mit $M_\text{J}=-\sfrac{1}{2}$ angeregt werden. Durch spontane Emission
aus dem angeregten Zustand, werden aber beide Grundzustände bevölkert. Das energieärmere Niveau
wird fortlaufend durch Einstrahlung von Licht geleert, das energetisch höhere mit
$M_\text{J}=+\sfrac{1}{2}$, aus dem keine Elektronen angeregt werden können, wird gesättigt.
Als Resultat steigt die Transmission mit der Zeit asymptotisch gegen den Wert 1, da keine
Elektronen mehr verfügbar sind, die angeregt werden können. Durch induzierte Emission können
höhere Niveaus durch Einstrahlen der passenden Energie wieder entvölkert werden. Stimmt die
eingestrahlte Energie mit der Energiedifferenz überein, wird ein Photon gleicher Energie
emittiert und das Atom befindet sich in einem energetisch niedrigeren Zustand.

\subsection{Messung der Zeeman-Aufspaltung}

Ein Elektron im angeregten Zustand kann durch spontane oder durch induzierte Emission von einem
Photon in seinen Grundzustand zurückkehren. Spontane Emission geschieht ohne äußere Einflüsse,
bei der induzierten Emission wird ein Photon eingestraht und und ein zweites Photon mit gleicher
Energie, Ausbreitungsrichtung und Polarisation wird emittiert.
Die Übergangswahrscheinlichkeit für spontane Emission ist proportional zu der dritten Potenz
der Frequenz der absorbierten und emittierten Photonen, während sie bei der induzierte Emission
in diesem Fall als konstant angenommen werden kann. Spontane Emission ist bei höheren Energien
wahrscheinlicher, die Energiedifferenz zwischen den hier betrachteten Zeeman-Niveaus in Rubidium
liegen jedoch im MHz-Bereich, in dem induzierte Emission den dominierenden Effekt darstellt.
Die Zeeman-Aufspaltung tritt nur bei angelgtem Magnetfeld auf, sodass auch nur bei externem
Magnetfeld das optische Pumpen stattfinden kann. Wird ein Magnetfeld angelegt, dass das
Erdmagnetfeld gerade ausgleicht, findet keine Aufspaltung statt, es tritt keine Inversion auf
und die Transmission sinkt. Ein solcher Einbruch in der Intensität des Lichts tritt
auch auf, wenn ein frequenzvariables Hochfrequenzfeld an die Dampfzelle angelegt wird. Das
Magnetfeld wird variiert und sobald es den Wert
\begin{equation}
  B_\text{m} = \frac{4\pi m_0}{\text{e}_0g_\text{J}}\nu
\end{equation}
erreicht, setzt induzierte Emission ein. Die Inversion wird aufgehoben, das eingestrahlte Licht
kann wieder absorbiert werden und die Transmission fällt ab. Die geschieht jeweils für die
beiden Rubidiumisotope bei unterschiedlichen Magnetfeldern.

\subsection{Quadratischer Zeeman-Effekt}

Wird die magnetische Flussdichte vergrößert, müssen bei der Berechnung der Übergangsenergie Terme höherer Ordnung von $B$
berücksichtigt werden, da nun die Spin-Bahn-Wechselwirkung relevant wird. Die Zeeman-Übergänge haben eine unterschiedliche
Energie und sind abhängig von $M_\text{F}$. Der Übergang von einem Zustand $M_\text{F}$ zu $M_\text{F}-1$ mit einer
Hyperfeinstrukturaufspaltung $\Delta E_\text{Hy}$ wird durch die Breit-Rabi-Formel beschrieben:
\begin{equation}
  U_\text{HF} = g_\text{F}\mu_\text{B}B + {g_\text{F}}^2{\mu_\text{B}}^2B^2\frac{1-2M_\text{F}}{\Delta E_\text{Hy}}
\end{equation}

\subsection{Transiente Effekte}

Wird das Hochfrequenzfeld schnell ein- und ausgeschaltet, präzediert der Kernspin um das effektive Magnetfeld. Die
Lamor-Frequenz $\\nu = \gamma B_\text{RF}$ mit dem gyromagnetischen Verhältnis $\gamma = g_\text{f}\frac{\mu_ß}{\text{h}}$
ist vom Lande-Faktor abhängig und somit für die beiden Rubidiumisotope verschieden. Das Verhältnis der Relaxationsperioden
$T = 1/\gamma B_\text{RF}$ ist dann
\begin{equation}
  \frac{T_{87}}{T_{85}} = \frac{\gamma_{85}}{\gamma_{87}}.
\end{equation}
