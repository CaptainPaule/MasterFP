\section{Durchf\"{u}hrung}

Die Myonen, die gemessen werden sollen, entstehen größtenteils aus Pionzerfällen in der oberen Atmosphäre. Aufgrund ihrer relativistischen Energie erreichen sie den Erdboden. Durch Wechselwirkung mit Materie geben sie einen Teil ihrer kinetischen Energie ab. Bei Durchgang durch einen Szintillator in einem Edelstahltank regt die abgegebene Energie das Szintillatormaterial an, sodass bei der Rückkehr in den Grundzustand Photonen im kurzwelligen sichtbaren bis UV-Bereich emittiert werden. Diese Photonen werden mit zwei Sekundärelektronenvervielfachern (SEV) detektiert, die an den Enden des Tanks angebracht sind. Niederenergetische Myonen können innerhalb des Detektionsvolumen in ein Elektron zerfallen, welches ebenfalls durch einen Lichtblitz ein Signal auslöst. Der zeitliche Abstand zwischen dem Myon- und dem Elektronsignal ist dann die Lebensdauer des Myons im Tank.

Um nicht den zeitlichen Abstand verschiedener Myonen zu messen, sondern die Lebensdauer zerfallender Myonen, wird
