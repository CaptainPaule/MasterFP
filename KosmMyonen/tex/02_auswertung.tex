\section{Auswertung}
In diesem Abschnitt wird die Lebensdauer komischer Myonen aus den Messdaten gewonnen.
Um dieses Ziel zu erreichen, ist es erforderlich die Messapparatur zu kalbibrieren. Außerdem
ist es erforderlich aus der Variation der Verzögerungsleitungen die 
Zeitauflösung der Messapparatur zu ermitteln. Alle Fehler werden im Folgenden mit Hilfe
des \texttt{python}-Paktes \texttt{uncertainties} brechnet, welches eine automatische
Gauß'sche Fehlerfortpflanzung bereitstellt.

\subsection{Kalibrierung der Messapparatur}