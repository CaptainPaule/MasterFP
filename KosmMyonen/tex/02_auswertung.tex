\section{Auswertung}
\label{sec:evaluation}
In diesem Abschnitt wird die Lebensdauer komischer Myonen aus den Messdaten gewonnen.
Um dieses Ziel zu erreichen, ist es erforderlich die Messapparatur zu kalbibrieren. Außerdem
ist es erforderlich aus der Variation der Verzögerungsleitungen die 
Zeitauflösung der Messapparatur zu ermitteln. Alle Fehler werden im Folgenden mit Hilfe
des \texttt{python}-Paktes \texttt{uncertainties} brechnet, welches eine automatische
Gauß'sche Fehlerfortpflanzung bereitstellt.

\subsection{Kalibrierung der Messapparatur}
\label{subsec:calibration}
Die Daten über die Lebensdauer wird mit Hilfe eines Vielkanalanalysators (VKA) gewonnen.
Der VKA teilt die gemessenen Lebensdauern in \num{512} Kanäle (Bins) eines Histogrammes ein.
Dabei ist zu berücksichtigen, das der Zusammenghang zwischen gemessener Lebensdauer und
Kanal linear zusammenhängt.

Somit ist es Sinnvoll dieses Zusammenhang über eine lineare Funktion

\begin{equation}
t(c) = A \cdot t + B
\end{equation}

anzunehmen. Die Software des VKA \texttt{Maestro V6.06} führt die Ausgleichrechnung von selbst aus
und liefert

\begin{align*}
	a &= \SI{-2.32}{\micro \second} \\
	\text{und} \quad b &= \SI[per-mode=fraction]{8.86}{\micro \second \per {Kanal}} \,.
\end{*align}

Im der weiteren Auswertung ist lediglich der Prameter $b$ von Interesse.

\subsection{Zeitauflösung der Messapparatur}
\label{subsec:timeresolution}
Um verschiedene Signallaufzeiten zwischen den SEV's und der Koinzidenz ausgleichen zu können, werden Verzögerungsleitunge
benutzt um weitere Verzögerungen in das System einbringen zu können. Durch Messung der Myonen-Zählrate unter Variation der Verzögerungszeiten lässt
sich die Auflösungszeit $\Delta t_\text{K}$ der Koinzidenzschaltung als Breite einer Gaußglocke bestimmen.


\subsection{Untergrundrate}
\label{subsec:underground}

\subsection{Bestimmung der Lebensdauer}
\label{subsec:livetime}
