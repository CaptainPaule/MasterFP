\section{Diskussion}

Bei der Amplitudenmodulation mit Trägerabstrahlung ist die Diskrepanz des Modulationgrades zwischen den unterschiedlichen Berechnungsarten mit 68\% zu groß, als dass sie mit Messungenauigkeiten oder Schwierigkeiten beim Ablesen erklärt werden könnte. Die deutet darauf hin, dass bei einer der Methoden ein Fehler bei der Berechnung vorgekommen ist. 

Gleiches ist vermutlich die Erklärung für einen Modulationsfaktor von $m = 51.7$ bei der Frequenzmodulation. Dieser Wert ist zu groß, als dass er bei den eingestellten Frequenz- und Amplitudenwerten der Träger- und Modulationsschwingung realistisch sein kann.

Dass ein Fehler in der Berechnung der Werte passiert ist, wird durch Signalbilder klar. Bei der Amplitudenmodulation ohne Trägerabstrahlung ist eine Schwebung zu sehen und im Ferquenzbild ist die Trägerfrequenz gegenüber der Modulationsfrequenz unterdrückt. Auch die Proportionalität zwischen Gleichspannung und Phase der Wechselspannung am Ringmodulator kann durch die kleinen Fehler von 2--5\% bei Durchführung einer linearen Ausgleichsrechnung gezeigt werden.

Die Demodulation von sowohl amplituden- als auch frequenzmodulierten Signalen funktioniert und Erhalt der Signalform.

\printbibliography
