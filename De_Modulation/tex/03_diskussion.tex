\section{Diskussion}

An den Oszilloskopaufnahmen und den Bildern des Frequenzanalysators ist zu erkennen, dass die Amplitudenmodulation ohne Trägerabstrahlung prinzipiell funktioniert. Dies ist an der Schwebung im modulierten Signal zu erkennen; die Trägerunterdrückung ist nicht perfekt, es ist immernoch ein Peak bei der Trägerfrequenz in der Aufnahme des Frequenzanalysators zu sehen.

Bei der Amplitudenmodulation mit Trägerabstrahlung ist die Diskrepanz des Modulationgrades zwischen den unterschiedlichen Berechnungsarten mit 68\% zu groß, als dass sie mit Messungenauigkeiten oder Schwierigkeiten beim Ablesen erklärt werden könnte. Dies kann dadurch erklärt werden, dass sich die Leistung der Seitenbänder auf unendlich viele Oberwellen verteilt, was im Frequenzraum berücksichtigt wird, im Zeitraum jedoch nicht, da hier nicht zwischen den Oberwellen unterschieden wrden kann. Dies wird auch dadurch bestätigt, dass der Modulationsgrad aus der Bestimmung durch minimale und maximale Spannungsamplitude kleiner ausfällt, als durch die Bestimmung über die Leistungpegel im Frequenzspektrum.\par 

Bei der Frequenzmodulation wird ein Modulationsgrad von $m = 0.134 \pm 0.003$ bestimmt, was realistisch scheint. Hauptursache für Fehler sind hier die ungenaue Messung der Verschmierung $\Delta t$ und die Abweichungen der Trägerfrequenz $\omega_\text{T}$ vom eingestellten Wert.\par

Auch die Proportionalität zwischen Gleichspannung und Phase der Wechselspannung am Ringmodulator kann durch die kleinen Fehler von 2.8\% bzw. 5\% bei Durchführung einer linearen Ausgleichsrechnung gezeigt werden. Auch hier ist eine Fehlerquelle, dass die Trägerfrequenz nicht exakt eingestellt werden kann und das in den elektrischen Komponenten Effekte auftreten, die Phase, Frequenz und Laufzeit beeinflussen, die in diesem Rahmen jedoch nicht näher untersucht und beschrieben werden.

Die Demodulation von einem amplitudenmodulierten Signal mit einem Ringmodulator funktioniert sehr gut. Das Signal wird mit einer konstanten Phasenverschiebung und leicht verkleinerten Amplitude rekonstruiert.

Bei der Demodulation eines amplitudenmodulierten Signals mit einer Gleichrichterdiode wird die Amplitude des demodulierten Signals durch den Tiefpass stark reduziert. Weiterhin tritt durch die Modulation eine Verdopplung der Frequenz auf.

Auch bei der Demodulation eines frequenzmodulierten Signals wird die Amplitude des demodulierten Signals durch den Tiefpass stark reduziert. Es wird dennoch gezeigt, dass das Verfahren funktioniert, da die Frequenzmodulation erfolgreich in eine Amplitudenmodulation überführt wird.

\printbibliography

\FloatBarrier

\section{Anhang}

\begin{table}
\centering
\caption{Messwerte für den linearen Zusammenhang zwischen Gleichspannung und Phase der Wechselspannung.}
\begin{tabular}{ccc}
$\omega_T$/MHz &  $U$/mV &  $U_0$/mV \\
\midrule
         4.2 &  -138 &     496 \\
         4.4 &  -116 &     456 \\
         4.6 &   -86 &     400 \\
         4.8 &   -51 &     362 \\
         5.0 &   -16 &     348 \\
         5.2 &    20 &     388 \\
         5.4 &    52 &     446 \\
         5.6 &    89 &     488 \\
         5.8 &   116 &     509 \\
\end{tabular}
\label{tab:e}
\end{table}


\FloatBarrier
